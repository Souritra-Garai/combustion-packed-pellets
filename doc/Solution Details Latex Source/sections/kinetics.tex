\documentclass[../main.tex]{subfiles}

\begin{document}


\subsection{Physics}

For a model reaction

$$ A \longrightarrow B $$

\noindent
Define the rate of change in conversion $\eta$

\begin{equation}
    \omega = \frac{d\eta}{dt}
\end{equation}

\noindent
Following a first order reaction rate model

\begin{equation}
    \omega(T, \eta) = k(T) \cdot (1-\eta)
\end{equation}

\noindent
Using Arrhenius Rate Law

\begin{equation}
    k(T) = A \cdot \exp{\left(-\frac{E_a}{R_u T}\right)}
\end{equation}



\subsection{Discretization}

Discretizing first order time derivative using backward difference scheme

\begin{equation}
    \frac{d\eta}{dt} \approx \frac{\eta_i^{n} - \eta_i^{n-1}}{\Delta t}
\end{equation}

\noindent
Building implicit update equation for $\eta$

$$
\frac{\eta_i^{n} - \eta_i^{n-1}}{\Delta t} = k\left(T_i^n\right) \cdot \left( 1 - \eta_i^{n} \right)
$$
$$
\Rightarrow \left[ 1 + k\left( T_i^n \right)\Delta t \right] \cdot \eta_i^n = \eta_i^{n-1} + k\left(T_i^n\right) \Delta t
$$

\begin{equation}
    \Rightarrow \eta_i^n = \frac{\eta_i^{n-1} + k\left(T_i^n\right) \Delta t}{1 + k\left( T_i^n \right)\Delta t}
\end{equation}



\subsection{Combustion of Ni-coated Al Pellets}

Assuming the diffusion of Al across the Ni coating as the rate dominating step, burning time is given by

\begin{equation}
    t_b = \omega ^{-1} = \frac{r^2}{cD(T)}
\end{equation}

where $r$ is the radius of the core, $D$ is the diffusion coefficient, and $c$ is the burning time constant, assumed to be six. \\

\noindent
The diffusion coefficient as function of temperature is governed by

\begin{equation}
    D(T) = D_0 \cdot \exp{\left( - \frac{E_a}{R_u T} \right)}
\end{equation}

where $D_0 = 9.54 \times 10^{-8}\;m^2/s$ and $E_a = 26\times  10^3\;J/mol$. \\

\noindent
We assume at $\eta=0$,

\begin{equation}
    \omega \vert_{\eta=0} = \frac{cD}{r} = 1.76123 \times 10^{-2} \cdot \exp{\left(-\frac{26\times10^3}{R_u T}\right)}
\end{equation}

\noindent
Comparing with the Arrhenius Rate Law, $A = 1.76123 \times 10^{-2}$ and $E_a = 26\times  10^3$ in SI units. The change in enthalpy for this reaction is $\Delta H = - 118.4 \times 10^3 \; J/mol$.

\end{document}